\documentclass{tccv}
\usepackage[english]{babel}
\usepackage{multicol}
\usepackage{geometry}
\geometry{left=1cm,right=1cm}
\begin{document}

\part{Junjie Guan}
	
\begin{multicols}{2}

\personal
    {4545 8th Ave NE, Seattle, WA}
    {+1 (603) 277 1196}
    {gjj2684@gmail.com}

\section{Interested field \& Skills}
- Distributed System, Service Oriented Architecture	
- Java, Ruby on Rails, Python, C++



\section{Education}

\begin{yearlist}

\item[Computer Science, M.S.]{09/2013 -- 03/2015}
     {Dartmouth College}
     %{Hanover, US}

\item[Communication Engineering, B.S.]{09/2009 -- 06/2013}
     {Beijing University of Posts and Telecommunication}
     %{Seminario vescovile, Cremona}

\end{yearlist}






\end{multicols}















\section{Professional Experience}

\begin{eventlist}

\item{04/2016 -- Present}
     {Software Engineer @ Airbnb}
     {Service Framework}
     
     Work on service framework that enable other engineers to build standard microservices in Airbnb. Before that worked on scaling memcached with mcrouter and scaling redis with twemproxy.
     

\item{04/2015 -- 02/2016}
     {Software Engineer @ Amazon AWS}
     {AWS Lambda}
     
     1) Worked on function versioning, a demanding feature to manage version and alias of Lambda function.  2) Implemented a test account pool service to avoid tests resource conflicts, which makes test parallelization possible. 3) Implemented a map-reduce-base analyser, which reduces analyzing terabytes of data from several days to a few hours. 4) Built more fine grain metrics for the service.

\item{06/2014 -- 08/2014}
     {Software Engineer Intern @ Amazon AWS}
     {Builder Tools, Package Builder Services}
     
    1) Implemented a feature to automize the process of pruning and unpruning broken package. 2) Implemented package build history API.
     
\item{09/2012 -- 02/2013}
     {Research Assistant @ NetLab, Tsinghua University}
     {DCloud: Deadline Guaranteed Cloud Computing}
     
     1) Designed a new datacenter resource allocation mechanism, by introducing job deadline when scheduling. 2) Wrote a resource allocation simulation program, and built a linear model to compute optimal results using IBM Cplex. 3) Validated the efficiency in a 16-machine cluster running Hadoop.
     \\\\
     Increase job throughput by 30$\scriptsize{\sim}$50\% comparing to state-of-art. Submitted to ACM SIGCOMM 2013.
     \\\\
     Publication: Li, D., Chen, C., \textbf{Guan, J.}, Zhang, Y., Zhu, J., \& Yu, R. DCloud: Deadline-aware Resource Allocation for Cloud Computing Jobs. Accepted, IEEE Transactions on Parallel and Distributed Systems 2015
     \\\\
     Publication: Dan Li, Jing Zhu, Jianping Wu, \textbf{Junjie, Guan}, Guaranteeing Heterogeneous Bandwidth Demand in Multi-tenant Data Center Networks. Accepted by IEEE/ACM Transactions on Networking ‘2014

     
     
     
     
     
     
    
    
    
    
    
    
    
    
    
    
    
    
    
    
    
   





\item{08/2011 -- 07/2012}
     {Software Developer @ Innovation Project Center of BUPT}
     {SIDES: Scalable Intelligent Distributed Emergency System}
     
     1) Implemented a decentralized self-organize wireless-based emergency system that generates real-time evacuation strategy to save lives in fire hazard. 2) Designed the learning protocol for distributed sensor nodes, And implemented in Zigbee.
\\\\
Shorten evacuation by 50\% comparing to normal strategy, Won first prize as National Class Innovation Project.
\\\\
Publication: \textbf{Junjie, Guan}, Yanyi Wu, Jinming Ma, Tao Li, Chunlei Xie, Yuli Mo SIDES: Scalable Intelligent Distributed Emergency System. Accepted by IEEE Network Infrastructure and Digital Content 2012
     
\end{eventlist}









































\if
\section{Professional skills}

\begin{factlist}

\item{Good level}
     {C++, Html(5), CSS(3), Javascript, jQuery, \LaTeX, Photoshop, Matlab}

\item{Intermediate}
     {Git, Python, Java, PHP, C\#, Bash shell script, MySQL, SqlServer, IBM Cplex Optimizer, Wireshark, AutoIt, Hadoop}

\item{Basic level}
     {Nodejs, Assembly, VHDL, NS2}

\end{factlist}

\fi











\end{document}
