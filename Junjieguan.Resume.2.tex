\documentclass{tccv}
\usepackage[english]{babel}
\usepackage{multicol}
\usepackage{geometry}
\geometry{left=1cm,right=1cm}
\begin{document}

\part{Junjie Guan (Jack)}
	
\begin{multicols}{2}

\personal
    {4545 8th Ave NE, Seattle, WA}
    {+1 (603) 277 1196}
    {gjj2684@gmail.com}

\section{Interested field \& Skills}
- Distributed System, Service Oriented Architecture	
- Java, Python, C++



\section{Education}

\begin{yearlist}

\item[Computer Science, M.S.]{2013 -- 2015}
     {Dartmouth College}
     %{Hanover, US}

\item[Communication Engineering, B.S.]{2009 -- 2013}
     {Beijing University of Posts and Telecommunication}
     %{Seminario vescovile, Cremona}

\end{yearlist}






\end{multicols}















\section{Professional Experience}

\begin{eventlist}

\item{Apr 2015 -- Present}
     {SDE @ Amazon AWS}
     {AWS Lambda}

     \quad1) Lambda function versioning. A  demanding feature that affect almost almost all the APIs and underneath function management architecture.
     
     \quad2) Implement a test account pool service. This avoids tests resource conflicts and unexpected account state change. Less mysterious test failures. Also it makes test parallelization possible and cuts hours of tests in half.
     
     \quad3) Implement a map reduce log analysis tool, which reduces analysing terabytes of data from serveral days, to a few hours.

     \quad4) Build fine grain metrics, monitors upon the service, along with a dashboard. Reduce uncaught events and more easy to pinpoint problem.


\item{Jun 2014 -- Aug 2014}
     {SDE Intern @ Amazon AWS}
     {Builder Tools, Package Builder Services}
     
     \quad1) Implement the Auto-Unprune feature to relief customer from tedious work with broken packages.
     
     \quad2) Implement pacakge build history API as strech project.
     
     
\item{Sep 2012 -- Feb 2013}
     {Research Assistant @ NetLab, Tsinghua University}
     {DCloud: Deadline Guaranteed Cloud Computing}
     
     \quad1) Design a new datacenter resource allocation mechanism, by considering deadline into job scheduling;
     
     \quad2) Build thousands of lines of C++ simulations program;
     
     \quad3) Implementation on a 16-machine cluster, writing controling  program with Python and Bash shell;
     
     \quad4) Validating efficiency with Hadoop application using; Java, 
     
     \quad5) Try compare with optimal solution using IBM Cplex;
	\\\\
     Increase job throughput by 30$\scriptsize{\sim}$50\% comparing to state-of-art, submitted to ACM SIGCOMM 2013.
     \\\\
     Publication: Li, D., Chen, C., \textbf{Guan, J.}, Zhang, Y., Zhu, J., \& Yu, R. DCloud: Deadline-aware Resource Allocation for Cloud Computing Jobs. Accepted, IEEE Transactions on Parallel and Distributed Systems
     \\\\
     Publication: Dan Li, Jing Zhu, Jianping Wu, \textbf{Junjie, Guan}, Guaranteeing Heterogeneous Bandwidth Demand in Multi-tenant Data Center Networks. Accepted by IEEE/ACM Transactions on Networking ‘2014

     
     
     
     
     
     
    
    
    
    
    
    
    
    
    
    
    
    
    
    
    
   





\item{Aug 2011 -- Jul 2012}
     {Software Developer @ Innovation Project Center of BUPT}
     {SIDES: Scalable Intelligent Distributed Emergency System}
     
     \quad1) Propose the idea of a decentralized self-organize wireless-based emergency system that generates real-time evaucation strategy to save lives in fire harzard, 
     
     \quad2) Design the the distributed protocol.
     
     \quad3) Conduct simulation on C++ program, and implement in Zigbee wireless nodes using C.
     
Shorten evacuation by 50\% comparing to normal strategy, Won first prize as National Class Innovation Project.
\\\\
Publication: \textbf{Junjie, Guan}, Yanyi Wu, Jinming Ma, Tao Li, Chunlei Xie, Yuli Mo SIDES: Scalable Intelligent Distributed Emergency System, accepted by IC-NIDC’2012
     
\end{eventlist}









































\if
\section{Professional skills}

\begin{factlist}

\item{Good level}
     {C++, Html(5), CSS(3), Javascript, jQuery, \LaTeX, Photoshop, Matlab}

\item{Intermediate}
     {Git, Python, Java, PHP, C\#, Bash shell script, MySQL, SqlServer, IBM Cplex Optimizer, Wireshark, AutoIt, Hadoop}

\item{Basic level}
     {Nodejs, Assembly, VHDL, NS2}

\end{factlist}

\fi











\end{document}
