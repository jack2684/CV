\documentclass{tccv}
\usepackage[english]{babel}
\usepackage{multicol}
\usepackage{geometry}
\geometry{left=1cm,right=1cm}
\begin{document}

\part{Junjie Guan (Jack)}

\begin{multicols}{2}

\personal
    [www.gjjhomepage.sinaapp.com]
    {Dartmouth College, 6211 Hinman}
    {+1 (603) 277 1196}
    {gjj@cs.dartmouth.edu}

\section{Interested field}
Networking, Cloud, Distributed Computing, \newline Web Design, Web Development


\section{Education}

\begin{yearlist}

\item[Computer Science, M.S.]{2013 -- 2015}
     {Dartmouth College}
     %{Hanover, US}

\item[Communication Engineering, , B.S.]{2009 -- 2012}
     {Beijing University of Posts and Telecommunication}
     %{Seminario vescovile, Cremona}

\end{yearlist}

\end{multicols}















\section{Professional Experience}

\begin{eventlist}

\item{Jan 2014 -- Present}
     {Startup co-founder \& lead developer}
     {ArtxChange}
     
     A market platform for art exchanging, while also a fundraising platform for innovative, socially beneficial projects. By doing so the artists can build their reputation while making profits from their art work. 

\item{Sep 2013 -- Present}
     {Software Developer @ Tiltfactor, Dartmouth}
     {Metadata Project}
     
     \quad1) Implementing games, realizing the media interface for other web application such as flickr;
     
     \quad2) Helping database design, security enhancement, fix any bugs listed in bug tracker (Asana);
     
     \quad3) Cooperation with other programers around the world using Git.
     
     Granted from National Endowment for the Humanities (NEH), a open source, internet‐based system for augmenting access to archival records.
     
     
\item{Sep 2012 -- Feb 2013}
     {Research Assistant @ NetLab, Tsinghua University}
     {DCloud: Deadline Guaranteed Cloud Computing}
     
     \quad1) Design a new datacenter resource allocation mechanism, by considering deadline into job scheduling;
     
     \quad2) Build thousands of lines of C++ simulations program;
     
     \quad3) Implementation on a 16-machine cluster, writing controling  program with Python and Bash shell;
     
     \quad4) Validating efficiency with Hadoop application using; Java, 
     
     \quad5) Try compare with optimal solution using IBM Cplex;

     Increase job throughput by 30$\scriptsize{\sim}$50\% comparing to state-of-art, submitted to ACM SIGCOMM 2013.
     
\item{Aug 2011 -- Jul 2012}
     {Research Assistant @ Innovation Project Center of BUPT}
     {SIDES: Scalable Intelligent Distributed Emergency System}
     
     \quad1) Propose the idea of a decentralized self-organize wireless-based emergency system that generates real-time evaucation strategy to save lives in fire harzard, 
     
     \quad2) Design the the distributed protocol.
     
     \quad3) Conduct simulation on C++ program, and implement in Zigbee wireless nodes using C.
     
Shorten evacuation by 50\% comparing to normal strategy, Won first prize as National Class Innovation Project.

Publication: \textbf{Guan, Junjie}, Yanyi Wu, Jinming Ma, Tao Li, Chunlei Xie, Yuli Mo SIDES: Scalable Intelligent Distributed Emergency System, accepted by IC-NIDC’2012
     
\end{eventlist}




















\clearpage
\section{Project in Competition}
\begin{eventlist}

\item{Dec 2011 -- May 2012}
     {Team Leader @ Microsoft Imagine Cup 2012}
     {UpAround Social Platform}
     
     \quad1) Proposed the idea of a location-based social platform, and initiated a technique team of four.
     
     \quad2) Manage the product development and presentation.
     
     \quad3) As lead programmer, built a full-stack social platform on .Net framework. 
     
     \quad4) Design and develop a highly-interactive UI using javascript/jQuery, html5 and css3.
     
Gained the 3rd prize in Software Design China Zone (6\%).   

Youtube Url: \underline{http://goo.gl/jwzWQq}
       
\end{eventlist}

















\section{Independent work}


\begin{eventlist}

\item{Summer 2012}
     {Undergraduate @ Future Network Lab of BUPT}
     {Status-Based Content Sharing Mechanism for CCN}
     
	Reduce overheads and increase transfer rate by aournd 30\% in Content-Centric Network by pre-determining a host based on source status. Then built a experiment network based on CCNx using C. 

Publication: \textbf{Guan, Junjie}, Xudong Wang, Yu Xia, Tao Huang, Liang Wei Status-Based Content Sharing Mechanism for Content-centric Network, accepted by ICCT’2012

\item{Apr 2011 -- May 2011}
     {Undergraduate @ Internet Research Department Center of BUPT}
     {WAP Topology Property Analysis: Comparison With WWW}     
     
\quad1) Developed a multi-thread web crawler using Python. 

\quad2) Keep recording statistics for weeks using mySql. 

\quad3) Analyzed the topology characteristics of WAP, e.g. degree distribution and clustering coefficient.\newline

\end{eventlist}






















\if
\section{Professional skills}

\begin{factlist}

\item{Good level}
     {C++, Html(5), CSS(3), Javascript, jQuery, \LaTeX, Photoshop, Matlab}

\item{Intermediate}
     {Git, Python, Java, PHP, C\#, Bash shell script, MySQL, SqlServer, IBM Cplex Optimizer, Wireshark, AutoIt, Hadoop}

\item{Basic level}
     {Nodejs, Assembly, VHDL, NS2}

\end{factlist}

\fi











\end{document}
