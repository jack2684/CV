\documentclass{tccv}
\usepackage[english]{babel}
\usepackage{multicol}


\begin{document}

\part{Junjie Guan (Jack)}

\begin{multicols}{2}



\personal
    [www.gjjhomepage.sinaapp.com]
    {1 Sanborn Rd., Hanover, NH}
    {+1 (603) 277 1196}
    {gjj@cs.dartmouth.edu}

\section{Interested field}
Networking, Cloud, Distributed Computing, Web Design, Web Development


\section{Education}

\begin{yearlist}

\item[Computer Science, M.S.]{2013 -- 2015}
     {Dartmouth College}
     %{Hanover, US}

\item[Communication Engineering, , B.S.]{2009 -- 2012}
     {Beijing University of Posts and Telecommunication}
     %{Seminario vescovile, Cremona}

\end{yearlist}

\end{multicols}















\section{Work Experience}

\begin{eventlist}

\item{Aug 2013 -- Present}
     {Software Developer @ Tiltfactor, Dartmouth}
     {Metadata Game Project}
     
     \quad1) developing and testing a full-stack website on Yii framework; 
     
     \quad2) Database Design 
     
     \quad3) Cooperation with other programers using Asana and Git.
     
     Granted from National Endowment for the Humanities (NEH), a open source, internet‐based system for augmenting access to archival records.
     
     
\item{Sep 2012 -- Feb 2013}
     {Research Assistant @ NetLab, Tsinghua University}
     {DCloud: Deadline Guaranteed Cloud Computing}
     
     \quad1) Design a new datacenter resource allocation mechanism, by considering deadline into job scheduling
     
     \quad2) Build thousands of lines of C++ simulations program
     
     \quad3) Implementation on a 16-machine cluster, writing controling  program with Python and Bash shell
     
     \quad4) Validating efficiency with Hadoop application using Java, 
     
     \quad5) Try compare with optimal solution using IBM Cplex.

     Increase job throughput by 30$\scriptsize{\sim}$50\% comparing to state-of-art, submitted to ACM SIGCOMM 2013 as second author.
     
\item{Aug 2011 -- Jul 2012}
     {Team Leader @ Innovation Project Center of BUPT}
     {SIDES: Scalable Intelligent Distributed Emergency System}
     
     \quad1) Propose the idea of a decentralized self-organize wireless-based emergency system that generates real-time evaucation strategy to save lives in fire harzard, 
     
     \quad2) Design the the distributed protocol.
     
     \quad3) Conduct simulation on C++ program, and implement in Zigbee wireless nodes using C.
     
Shorten evacuation by around 50\% comparing to normal strategy, Won first prize as National Class Innovation Project.

Publication: \textbf{Guan, Junjie}, et al. "SIDES: Scalable Intelligent Distributed Emergency System." Network Infrastructure and Digital Content (IC-NIDC), 2012 3rd IEEE International Conference on. IEEE, 2012.     
     
\end{eventlist}





















\section{Porjects in competitions}


\begin{eventlist}

\item{Dec 2011 -- May 2012}
     {Team Leader @ Microsoft Imagine Cup 2012}
     {UpAround Social Platform}
     
     \quad1) Propose the idea of a location-based social platform, and initiate a technique team of 4.
     
     \quad2) Manage the product development and presentation.
     
     \quad3) As lead programmer, built a full-stack social platform on .Net framework. 
     
     \quad4) Design and develop a highly-interactive UI using javascript/jQuery, html5, css3.
     
Gained the 3rd prize in Software Design China Zone (6\%).   

Url: \underline{http://www.youtube.com/watch?v=b8UHGPxJFyQ}
       
\end{eventlist}

















\section{Independent work}


\begin{eventlist}

\item{Aug 2011 -- Feb 2013}
     {Undergraduate @ Future Network Lab of BUPT}
     {Status-Based Content Sharing Mechanism for CCN}
     
	Reduce overheads and increase transfer rate by aournd 30\% in Content-Centric Network by pre-determining a host based on source status. Then built a experiment network based on CCNx using C. 

Publication: \textbf{Guan, Junjie}, et al. "Status-Based Content Sharing Mechanism for Content-Centric Network." Communication Technology (ICCT), 2012 IEEE 14th International Conference on. IEEE, 2012. 

\item{Apr 2011 -- May 2011}
     {Undergraduate @ Internet Research Department Center of BUPT}
     {WAP Topology Property Analysis: Comparison With WWW}     
     
\quad1) Developed a multi-thread web crawler using Python. 

\quad2) Keep recording statistics for weeks using mySql. 

\quad3) Analyzed the topology characteristics of WAP ,e.g. degree distribution and clustering coefficient.\newline

\end{eventlist}














\section{Academic projects}

\begin{eventlist}
\if
\item{Current}
     {@Dartmouth}
     {}
     
     \textbf{ADTCP: ADptation TCP}\newline
     A research-oriented project. In order to address the problem of congestion and TCP incast in datacenter network, state-of-art usualy design flow control in swith side, which is too ahead of time and hard to realize. My goal is utilize the intelligence of edge and keep network simple, by implementing flow control on the edge/server side.
\fi

\item{Before}
     {@BUPT}
     {}     
     
     \textbf{The Construction of Campus Network}\newline
     Design and construct a campus network from bottom to top layer, and configured enterprise level routers and switches for routing table, Vlan, ACL.\newline

     \textbf{Interactive Smith Chart - A Html5 Web App}\newline
     A nicely designed and useful Smith chart using html5 canvas and jQuery, which is a graphical aid or nomogram designed for electrical and electronics engineers.\newline
     
     Url: \underline{http://jacksmithchart.sinaapp.com/}\newline

     \textbf{A Course Sign-up Robot Program}\newline
     A robot program written by AutoIt, which is capable of auto-finishing course sign-up instantly.

\end{eventlist}












\section{Professional skills}

\begin{factlist}

\item{Good level}
     {C++, Html(5), CSS(3), Javascript, jQuery, \LaTeX, Photoshop, Matlab}

\item{Intermediate}
     {Git, Python, Java, PHP, C\#, Bash shell script, MySQL, SqlServer, IBM Cplex Optimizer, Wireshark, AutoIt, Hadoop}

\item{Basic level}
     {Nodejs, Assembly, VHDL, NS2}

\end{factlist}
















\end{document}
