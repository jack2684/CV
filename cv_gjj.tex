\documentclass{tccv}
\usepackage[english]{babel}


\begin{document}

\part{Junjie Guan (Jack)}


\personal
    [www.gjjhomepage.sinaapp.com]
    {1 Sanborn Rd., Hanover, NH}
    {+1 (603) 277 1196}
    {gjj@cs.dartmouth.edu}

\section{Education}

\begin{yearlist}

\item[Computer Science, M.S.]{2013 -- 2015}
     {Dartmouth College}
     %{Hanover, US}

\item[Communication Engineering, , B.S.]{2009 -- 2012}
     {Beijing University of Posts and Telecommunication}
     %{Seminario vescovile, Cremona}

\end{yearlist}

\section{Interested field}
Networking, Distributed Computing, Cloud, Web Design, Web Development


















\section{Pro \& Research Experience}

\begin{eventlist}

\item{Aug 2013 -- Present}
     {Programmer @ Tiltfactor, Dartmouth}
     {Metadata Game Project}
     
     Granted from National Endowment for the Humanities (NEH), a open source, internet‐based system for augmenting access to archival records. Responsible for 1) developing and testing a full-stack website on Yii framework; 2) Database Design 3) Cooperation with other programers using Asana and Git. \newline
     
\item{Sep 2012 -- Feb 2013}
     {Lead Researcher @ NetLab, Tsinghua University}
     {DCloud: Deadline Guaranteed Cloud Computing}
     
     Submitted to ACM SIGCOMM 2013. As lead researher I 1) incorporate deadline, flexible cloud resource allocation into job scheduling of datacenter. 2) Build thousands of lines of C++ simulations program, 3) conduct implementation on a 16 machine cluster, writing controling and sampling program using Python and Bash shell, validating efficiency with Hadoop application using Java, 4) Compare with optimal solution using IBM Cplex. \newline
     
\item{Apr 2012 -- Aug 2012}
     {Researcher @ NetLab, Tsinghua University}
     {Towards Bandwidth Sharing in Multi-tenants Datacenters}
     
     Submitted to ACM Transaction on Networking. I was responsible for the design of an allocation algorithm for heterogeneous matrix resource request, by finding the approximate optimal solution from a $O(2^n)$ space. Develop simulation with C++. \newline
     
\item{Dec 2011 -- May 2012}
     {Team Leader @ Microsoft Imagine Cup 2012}
     {UpAround Social Platform}
     
     A location-based social platform. I initiate a technique team of 4 and gained the 3rd prize in Software Design China Zone (6\%). I 1) manage the product development and presentation, 2) as lead programmer, built a full-stack high-interactive social platform on .Net framework. Design and develop UI using javascript/jQuery, html5, css3. \newline
         
\item{Aug 2011 -- Jul 2012}
     {Team Leader @ Innovation Project Center of BUPT}
     {SIDES: Scalable Intelligent Distributed Emergency System}
     
     A decentralized self-organize wireless-based emergency system that generates real-time evaucation strategy to save lives in fire harzard, won first prize award in National Class Innovation Project. As team leader of 5 I design the the distributed protocol, conduct simulation on C++ program, and implement in Zigbee wireless nodes using C.\newline
     
\item{Aug 2011 -- Feb 2013}
     {Lead Researcher @ Future Network Lab of BUPT}
     {Status-Based Content Sharing Mechanism for CCN}
     
     To address the overhead problem in current content-centric network, I designed a application layer mechanism to reduce content-centric network overheads by pre-determining a source host for content caching based on source status. Then built a experiment network based on CCNx using C.\newline
     

\item{Apr 2011 -- May 2011}
     {Programmer @ Internet Research Department Center of BUPT}
     {WAP Topology Property Analysis: Comparison With WWW}     
     
1) Developed a multi-thread web crawler using Python. 2) Recording statistics using mySql from a week of running. 3) Analyzed the topology characteristics of WAP ,e.g. degree distribution and clustering coefficient.\newline
     
\end{eventlist}















\section{Academic projects}

\begin{eventlist}

\item{Current}
     {@Dartmouth}
     {}
     
     \textbf{ADTCP: ADptation TCP}\newline
     A research-oriented project. In order to address the problem of congestion and TCP incast in datacenter network, state-of-art usualy design flow control in swith side, which is too ahead of time and hard to realize. My goal is utilize the intelligence of edge and keep network simple, by implementing flow control on the edge/server side.

\item{Before}
     {@BUPT}
     {}
     
     \textbf{Network Allocation based on MapReudce Model}\newline
     In this work,I adopt matrix model against current coarse-grained hose model, and overcome the complexity issue of matrix model by incorporating the special network pattern underneath MapReduce. I put forward a new demand abstract and corresponding resource allocation algorithm. Simulation work has been done for theoretical validation. I also realize the functional module in Hadoop.\newline
     
     
     \textbf{The Construction of Campus Network}\newline
     Design and construction of a campus network, which is fully functional from bottom to top layer. I designed network architecture, and configured enterprise level router, switch, such as routing protocol, network address assignment, VLan, Access Control List.\newline

     \textbf{Interactive Smith Chart - A Html5 Web App}\newline
     A nicely designed and useful Smith chart using html5 canvas and jQuery, which is a graphical aid or nomogram designed for electrical and electronics engineers.\newline

     \textbf{A Course Sign-up Robot Program}\newline
     A robot program written by AutoIt, which is capable of auto-finishing course sign-up instantly.\newline\newline\newline\newline\newline\newline\newline\newline\newline\newline\newline\newline\newline\newline\newline\newline\newline\newline\newline\newline

\end{eventlist}














\section{Publications}

\textbf{Guan, Junjie}, et al. "SIDES: Scalable Intelligent Distributed Emergency System." Network Infrastructure and Digital Content (IC-NIDC), 2012 3rd IEEE International Conference on. IEEE, 2012.\newline

\textbf{Guan, Junjie}, et al. "Status-Based Content Sharing Mechanism for Content-Centric Network." Communication Technology (ICCT), 2012 IEEE 14th International Conference on. IEEE, 2012. 









\section{Professional skills}

\begin{factlist}

\item{Good level}
     {C++, C\#, Python, Bash shell script, Html(5), CSS(3), Javascript, .Net, jQuery, \LaTeX, Photoshop, Matlab, Hadoop, Jason, xml}

\item{Intermediate}
     {Git, MySQL, SqlServer, IBM Cplex Optimizer, VHDL, Java, PHP, Wireshark, Multisim, Protel, Linux, AutoIt}

\item{Basic level}
     {Nodejs, Assembly, Yii, NS2}

\end{factlist}









\section{Communication skills}

\begin{factlist}
\item{Mandarin}{Native speaker}
\item{Cantonese}{Native speaker}
\item{English}{Good}	
\end{factlist}










\end{document}
